\documentclass[11pt]{article}

\setlength{\oddsidemargin}{0.0in}
\setlength{\evensidemargin}{0.0in}
\setlength{\textwidth}{6.25in}
\setlength{\topmargin}{-0.4in}
\setlength{\textheight}{8.5in}
\setlength{\parindent}{0em}
\setlength{\parskip}{1.5ex}

\usepackage{amssymb, amsmath, amsfonts}
\usepackage{mathpazo} % palatino math fonts
%\usepackage{palatino}
%\usepackage{utopia}
\usepackage{latexsym}
\usepackage{verbatim}
\usepackage{graphics}

\usepackage[hyperindex]{hyperref}
\usepackage{url}
\urlstyle{sf}

\newcommand{\marginquote}[2]{\marginpar[\footnotesize \raggedleft {\em #1} \\ 
                                        \scriptsize #2]
                            {\footnotesize \raggedright {\em #1} \\ \scriptsize #2}}
\newcommand{\sidequote}{\marginquote}
\newcommand{\shortsection}[1]{\textbf{#1. }}

\newenvironment{enumtight}{\vspace*{-2ex}\begin{enumerate}\setlength{\itemsep}{0cm} \setlength{\parskip}{0cm}\setlength{\parskip}{0cm}\setlength{\parsep}{3pt}\setlength{\topsep}{3pt}\setlength{\partopsep}{0pt}}{\end{enumerate}\vspace*{-1.3ex}}
\newenvironment{itemtight}{\vspace*{-2ex}\begin{itemize}\setlength{\itemsep}{0cm} \setlength{\parskip}{0cm}\setlength{\parsep}{3pt} \setlength{\topsep}{3pt}\setlength{\partopsep}{0pt}}{\end{itemize}\vspace*{-1.3ex}}

\newenvironment{smallquote}{\vspace*{-2ex}\begin{list}{}{%
  \setlength\rightmargin{1.5em}\setlength\leftmargin{1.5em}\setlength\labelwidth{0pt}\setlength\itemindent{0pt}}\item[]}{\end{list}\vspace*{-1.3ex}}
\newcommand{\nonterminal}[1]{{\sl #1}}
\newcommand{\terminal}[1] {{\textbf{#1}}}
\newcommand{\produces}{$\rightarrow$}
\newenvironment{bnfgrammar}{\begin{quote}\begin{tabbing} \hspace*{3em}\=\ \produces\quad\= \kill}{\end{tabbing}\end{quote}}
\newenvironment{bnfgrammarm}[1]{\begin{quote}\begin{tabbing} #1\qquad\=\ \produces\quad\= \kill}{\end{tabbing}\end{quote}}
\newenvironment{smallbnfgrammar}{\begin{quote}\begin{tabbing} \hspace*{3em}\=\ \produces\quad\= \kill}{\end{tabbing}\end{quote}}
\newcommand{\bnfrule}[2]{\hfill\nonterminal{#1}\>\produces\>#2\\}

\newcounter{problemno}
\newcommand{\problem}[1]{
   \stepcounter{problemno}
	 {\bf Problem \theproblemno: #1.}}
	 
\newcommand{\handout}[3]{
   \renewcommand{\thepage}{#1-\arabic{page}}
   \noindent
   \begin{center}
\scalebox{0.2}{\includegraphics{forthare.png}}
\vspace{5mm}
   \framebox{
      \vbox{
   {\bf RANDOM EFFECTS MODELLING (STS 505)} 
       \vspace{7mm}
       \hbox to 5.78in { {\Large \hfill #2  \hfill} }
       \vspace{-5mm}
       \hbox to 5.78in { { \hfill {\bf #3}} }
      }
   }
   \end{center}
   \vspace*{4mm}
}


\newcommand{\fillin}[1]{\vspace*{#1}}

%   % \begin{center}
%   \ \\ \framebox{
%      \vbox{
%    \hbox to 6.0in {\vspace{#1}}}
%    }
%   %\end{center}
%}

\newcommand{\fillinshort}[1]{
   \framebox{
      \vbox{
    \hbox to #1 {\vspace{4mm}}}}
}

\newcommand{\fillinline}[1]{
   \framebox{
      \vbox{
    \hbox to #1 {\vspace{8mm}}}}
}

\renewcommand\theenumi{\alph{enumi}}
\newcommand{\answer}[1]{\par \bigskip \begin{smallquote}\emph{Answer:} #1\end{smallquote}}

\begin{document}
\handout{}{\bf Assignment One}

{\bf INSTRUCTIONS.}  

{\bf Title:} {\bf Statistical Analysis Assignment} 

{\bf Objective:} The goal of this assignment is to analyse the relationship between study hours and exam scores using statistical methods. You will collect, organise, and interpret data to draw meaningful conclusions about the relationship between these variables.

{\bf Step 1: }Data Collection

{\bf Step 2:} Data Exploration


{\bf Submission:}
Prepare a written report that includes all the steps mentioned above. Your report should be well-organised, clearly written, and appropriately formatted. Include tables, graphs, and calculations where necessary. Make sure to provide a thorough explanation of your analysis and conclusions.

{\bf Note:} If you encounter any difficulties during the assignment, feel free to reach out for assistance. Good luck with your STS505 assignment!

\vspace*{1cm}

{\bf Lecturer} 

{\bf Dr. Azeez Adeboye}


\newpage

{\bf Question One} (25 marks)

Classify each of the following languages as:
(R) Regular, 
(CF) Context-Free, but not regular, 
(TD) Turing-Decidable, but not context-free, 
(TR) Turing-Recognizable, but not Turing-decidable,
or (None) None of the above.

For full credit, your answer should include a brief argument supporting your answer (but a detailed proof is not needed).

\begin{enumerate}

\item (5) The set of strings generated by the replacement grammar ($S$ is the start variable):
\begin{bnfgrammarm}{S}
\bnfrule{$S$}{$AS \mid SB$} 
\bnfrule{$A$}{$0 \mid S$} 
\bnfrule{$B$}{$1 \mid S$} 
\end{bnfgrammarm}

\answer{Your answer here}


\item (5) $R_{TM} = \{ \langle M, w\rangle \mid M $ is a description of a TM and $w$ is not accepted by $M$ $\}$

\answer{Your answer here}

\item (5) $POWER = \{ 1^{x}**1^{y}=1^{z} \mid x, y, z \in \mathcal{N}, z = x^y \}$

\answer{Your answer here}

\item (5) {\em UVA-WINS} = $\{ y \mid y $ is a four-digit string representing a year between 2000 and 2999 and UVa wins an NCAA soccer championship (Men's or Women's) in that year $\}$

\answer{Your answer here}


\end{enumerate}

\problem{Janus Machine} 
\[
\delta: Q \times \Gamma \times \Gamma \rightarrow Q \times \Gamma \times \{ \text{\bf L, R} \} \times \Gamma \times \{ \text{\bf L, R} \}
\]

\answer{Your answer here}

\problem{Fading Tape Machine}   addition of $K \in \mathcal{N}$, 
\[
\delta^{*}: (\Gamma, \mathcal{N})^{*} \times Q \times \mathcal{N} \times (\Gamma, \mathcal{N})^{*} \rightarrow (\Gamma, \mathcal{N})^{*} \times Q \times \mathcal{N} \times (\Gamma, \mathcal{N})^{*}
\]

A partial definition of $\delta^{*}$ is below (we have omitted the rules for dealing with the edge of the tape, but these are similar to the rules for a standard TM).

\begin{tabbing}
\hspace*{0.1in} \= \hspace*{0.1in} \= \hspace*{0.1in} \= \hspace*{0.1in} \= \kill 
$\forall u, v \in (\Gamma, \mathcal{N})^{*}, a, b \in \Gamma, n_i \in \mathcal{N}, q \in Q$:\\
\> if $q \in \left\{q_{accept}, q_{reject}\right\}$: \\
\> \> \> $\delta^{*}(u, q_{F}, n, v) = (u, q_{F}, n, v)$\\
\> else:\\
\> \> if $\delta(q, \sqcup) = (q_r, c, \textbf{L})$ and $n_s - n_b > K$ or $b = \sqcup$:\\
\> \> \> \> $\delta^{*}(u(a, n_a), q, n_s, (b, n_b)v) = \delta^{*}(u, q_r, n_s + 1, (a, n_a)(c, 0)v)$ \\
\> \> if $\delta(q, \sqcup) = (q_r, c, \textbf{R})$ and $n_s - n_b > K$ or $b = \sqcup$:\\
\> \> \> \> $\delta^{*}(u, q, n_s, (b, n_b)v) = \delta^{*}(u(c, 0), q_r, n_s + 1, v)$ \\
\> \> if $\delta(q, b) = (q_r, c, \textbf{L})$ and $n_s - n_b \le K$:\\
\> \> \> \> $\delta^{*}(u(a, n_a), q, n_s, (b, n_b)v) = \delta^{*}(u, q_r, n_s + 1, (a, n_a)(c, 0)v)$ \\
\> \> if $\delta(q, b) = (q_r, c, \textbf{R})$ and $n_s - n_b \le K$\\
\> \> \> \> $\delta^{*}(u, q, n_s, (b, n_b)v) = \delta^{*}(u(c, 0), q_r, n_s + 1, v)$\\
\end{tabbing}

\begin{enumerate}
\item (10) To define the computing model for the Fading Tape machine below.
\begin{quote}
A Fading Tape Machine $M = (Q, \Sigma, \Gamma, \delta, q_0, q_{accept}, q_{reject}, K)$ 
accepts a string $w \in \Sigma^{*}$ iff 
\[
\delta^{*}(\text{{\em initial configuration needed}}) = (\gamma_L, q_{accept}, \gamma_R)
\]
for some $\gamma_L, \gamma_R \in (\Gamma, \mathcal{N})^{*}$.
\end{quote}

\answer{Your answer here}

\item (10) Define precisely the power of a Fading Tape machine.  Support your answer with a very convincing argument.

\answer{Your answer here}

\end{enumerate}


\begin{center}

{\Large
		\begin{tabular}{|c|c|c|c|c|c|} \hline
      1 (20) & 2 (30) & 3 (20) & 4 (15) & 5 (20) & Total (105) \\ \hline
			\qquad\qquad  &\qquad\qquad   &\qquad\qquad   & \qquad\qquad  &\qquad\qquad   & \qquad\qquad  \\ \hline
		\end{tabular}
}
\end{center}

\end{document}


